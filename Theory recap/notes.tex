\documentclass{article}
% LTeX: enabled=false

\usepackage{amsfonts,amssymb,amsmath,amsthm, dsfont}

%% set-style letters
\def\AA{{\mathbb{A}}}
\def\BB{{\mathbb{B}}}
\def\CC{{\mathbb{C}}}
\def\DD{{\mathbb{D}}}
\def\EE{{\mathbb{E}}}
\def\FF{{\mathbb{F}}}
\def\GG{{\mathbb{G}}}
\def\HH{{\mathbb{H}}}
\def\II{{\mathbb{I}}}
\def\JJ{{\mathbb{J}}}
\def\KK{{\mathbb{K}}}
\def\LL{{\mathbb{L}}}
\def\MM{{\mathbb{M}}}
\def\NN{{\mathbb{N}}}
\def\OO{{\mathbb{O}}}
\def\PP{{\mathbb{P}}}
\def\QQ{{\mathbb{Q}}}
\def\RR{{\mathbb{R}}}
\def\SS{{\mathbb{S}}}
\def\TT{{\mathbb{T}}}
\def\UU{{\mathbb{U}}}
\def\VV{{\mathbb{V}}}
\def\WW{{\mathbb{W}}}
\def\XX{{\mathbb{X}}}
\def\YY{{\mathbb{Y}}}
\def\ZZ{{\mathbb{Z}}}

%% set-style numbers
\def\one{{\mathds{1}}}

%% calligraphic letters
\def\cA{{\mathcal{A}}}
\def\cB{{\mathcal{B}}}
\def\cC{{\mathcal{C}}}
\def\cD{{\mathcal{D}}}
\def\cE{{\mathcal{E}}}
\def\cF{{\mathcal{F}}}
\def\cG{{\mathcal{G}}}
\def\cH{{\mathcal{H}}}
\def\cI{{\mathcal{I}}}
\def\cJ{{\mathcal{J}}}
\def\cK{{\mathcal{K}}}
\def\cL{{\mathcal{L}}}
\def\cM{{\mathcal{M}}}
\def\cN{{\mathcal{N}}}
\def\cO{{\mathcal{O}}}
\def\cP{{\mathcal{P}}}
\def\cQ{{\mathcal{Q}}}
\def\cR{{\mathcal{R}}}
\def\cS{{\mathcal{S}}}
\def\cT{{\mathcal{T}}}
\def\cU{{\mathcal{U}}}
\def\cV{{\mathcal{V}}}
\def\cW{{\mathcal{W}}}
\def\cX{{\mathcal{X}}}
\def\cY{{\mathcal{Y}}}
\def\cZ{{\mathcal{Z}}}
\def\cKL{{\mathcal{KL}}}

%% bold letters
\def\bA{{\bf{A}}}
\def\bB{{\bf{B}}}
\def\bC{{\bf{C}}}
\def\bD{{\bf{D}}}
\def\bE{{\bf{E}}}
\def\bF{{\bf{F}}}
\def\bG{{\bf{G}}}
\def\bH{{\bf{H}}}
\def\bI{{\bf{I}}}
\def\bJ{{\bf{J}}}
\def\bK{{\bf{K}}}
\def\bL{{\bf{L}}}
\def\bM{{\bf{M}}}
\def\bN{{\bf{N}}}
\def\bO{{\bf{O}}}
\def\bP{{\bf{P}}}
\def\bQ{{\bf{Q}}}
\def\bR{{\bf{R}}}
\def\bS{{\bf{S}}}
\def\bT{{\bf{T}}}
\def\bU{{\bf{U}}}
\def\bV{{\bf{V}}}
\def\bW{{\bf{W}}}
\def\bX{{\bf{X}}}
\def\bY{{\bf{Y}}}
\def\bZ{{\bf{Z}}}
\def\ba{{\bf{a}}}
\def\bb{{\bf{b}}}
\def\bc{{\bf{c}}}
\def\bd{{\bf{d}}}
\def\be{{\bf{e}}}
\def\boldf{{\bf{f}}} %different
\def\bg{{\bf{g}}}
\def\bh{{\bf{h}}}
\def\bi{{\bf{i}}}
\def\bj{{\bf{j}}}
\def\bk{{\bf{k}}}
\def\bl{{\bf{l}}}
\def\bm{{\bf{m}}}
\def\bn{{\bf{n}}}
\def\bo{{\bf{o}}}
\def\bp{{\bf{p}}}
\def\bq{{\bf{q}}}
\def\br{{\bf{r}}}
\def\bs{{\bf{s}}}
\def\bt{{\bf{t}}}
\def\bu{{\bf{u}}}
\def\bv{{\bf{v}}}
\def\bw{{\bf{w}}}
\def\bx{{\bf{x}}}
\def\by{{\bf{y}}}
\def\bz{{\bf{z}}}

%% other symbols
\DeclareMathOperator{\1}{\mathbf{1}}
\DeclareMathOperator{\0}{\mathbf{0}}
\DeclareMathOperator{\Id}{I}
\newcommand{\td}{\mathfrak{t}} % discrete-time 

%% operators
\DeclareMathOperator{\col}{col}
\DeclareMathOperator{\diag}{diag}
\DeclareMathOperator{\blkdiag}{blkdiag}
\DeclareMathOperator{\rank}{rank}
\DeclareMathOperator{\dis}{d}
\DeclareMathOperator{\sat}{sat} 
\DeclareMathOperator{\convhull}{\textbf{co}}
\DeclareMathOperator{\argmin}{argmin}
\DeclareMathOperator{\argmax}{argmax}
\DeclareMathOperator{\spec}{spec}
\def\He#1{\texttt{\rm{He}}\left\{{#1}\right\}}
\DeclareMathOperator{\trace}{tr}
\newcommand{\Imag}{\mathrm{Im}}
\newcommand{\tr}{^\top}

%% shortcuts
\newcommand{\norm}[1]{\lvert #1\rvert}
\newcommand{\wnorm}[2]{\lvert #1\rvert^2_{#2}}
\newcommand{\pderiv}[2]{\dfrac{\partial #1}{\partial #2}}
\newcommand{\pdef}[1]{\SS_{\succ0}^{#1}}
\newcommand\psemidef[1]{\SS_{\succeq0}^{#1}}
\newcommand{\bmx}[1]{\left[\begin{matrix}#1\end{matrix}\right]}
\newcommand{\pmx}[1]{\left(\begin{matrix}#1\end{matrix}\right)}
\newcommand{\smallpmat}[1]{\left(\begin{smallmatrix} #1 \end{smallmatrix} \right)}
\newcommand{\smallqmat}[1]{\left[\begin{smallmatrix} #1 \end{smallmatrix} \right]}
\newcommand{\overbar}[1]{\mkern 1.5mu\overline{\mkern-1.5mu#1\mkern-1.5mu}\mkern 1.5mu}
\renewcommand{\underbar}[1]{\mkern 1mu\underline{\mkern-1mu#1\mkern-1mu}\mkern 1mu}

% LTeX: enabled=false

\usepackage{hyperref}
\usepackage{graphicx}
\usepackage{subcaption}
\usepackage{float}

% SCRIPTS FOR DOUBLE AND SINGLE IMAGE

% \begin{figure}[H]
%     \centering
%     \begin{subfigure}{0.4\textwidth}
%     \includegraphics[width=\textwidth]{}
%     \caption{}
%     \label{}
%     \end{subfigure}
%     \hfill
%     \begin{subfigure}{0.55\textwidth}
%     \includegraphics[width=\textwidth]{}
%     \caption{}
%     \label{}
%     \end{subfigure}
%     \caption{}
%     \label{}
% \end{figure}

% \begin{figure}[H]
%     \centering
%     \includegraphics[width=0.65\textwidth]{}
%     \caption{}
%     \label{}
% \end{figure}

\usepackage[margin=54pt]{geometry}
\usepackage{biblatex}
\addbibresource{biblio.bib}


\begin{document}

\date{Theory recap}
\author{Marco Sterlini}

\title{Title}
\maketitle

\section*{First works}
My contribution started by exploring the works \cite{css-paper} and \cite{css-extended}. The problem addressed is the design of an event-triggered control (ETC) for discrete-time linear-time-invariant (LTI) systems.
The aim was to decrease the computational load that follows the application of non-linear activation functions inside the neural network (NN) controller. Also thanks to previous works \cite{sophie-book-saturation}, \cite{arcak-first}, \cite{iqc-intro} several tools to treat these non-linearities have been studied and applied allowing the use of classical stability analysis methods to highly non-linear NN controllers. Lyapunov theory will be used to guarantee the stability of the NN controlled plant.

\subsection*{System under analysis}

The feedback system under examination is the following:

\begin{figure}[H]
    \centering
    \includegraphics[width=0.6\textwidth]{img/simple-etm}
    \label{}
\end{figure}

The NN controller is called $\pi_{ETM}$, the plant to stabilize is $F$, and it is described by the dynamics:

\begin{align*}
    &x(k + 1) = A_{F} x(k) + B_{F} u(k)\\
    \\
    &\hat{\omega}^{0}(k) = x(k),\\
    &\nu^{i}(k) = W^{i} \hat{\omega}^{i-1}(k) + b^{i},\ \ i \in \left\{ 1, \dots, l \right\},\\
    &\omega^{i}(k) = \text{sat}(\nu^{i}(k)),\\
    &u(k) = W^{l+1} \hat{\omega}^{l} (k) + b^{l+1}\\ 
\end{align*}

Where the first equation describes the plant $F$ as a discrete-time linear time-invariant system, the other equations describe the behavior of the NN controller: $\hat{\omega}^{i}$ and $\omega^{i}$ are respectively the last and current output of the $i^{th}$ layer, $\nu^{i}$ is the current input to the $i^{th}$ layer. The control input $u(k)$ is hence the output of the last layer of the NN, also the weights and the biases of the $i^{th}$ layer are indicated by $W^{i}$ and $b^{i}$. As per the activation function the saturation function was used and is defined by:
$$
   \text{sat}(\nu^{i}_{j}(k)) = \text{sign}(\nu^{i}_{j}(k)) min(|\nu^{i}_{j}(k)|, \bar{\nu}^{i}_{j})
$$  
For the $j^{th}$ neuron of the $i^{th}$ layer where $\bar{\nu}^{i}_{j}$ is the relative maximum limit value for the input signal. The activation function is applied element wise but a notation to isolate the non linearities is used:
$$
    \nu_{\phi} = \bmx{\nu^{1\tr}, \dots \nu^{l\tr}}\tr, \omega_{\phi} = \bmx{\omega^{1\tr}, \dots \omega^{l\tr}}\tr, \hat{\omega}_{\phi} = \bmx{\hat{\omega}^{1\tr}, \dots \hat{\omega}^{l\tr}}\tr \in \RR^{n_{\phi}}
$$
With $n_{\phi}$ the total number of neurons involved in the controller. It is then possible to express:
$$
    \text{sat}(v_{\phi}) = \bmx{\text{sat}(\nu^{1})\tr, \dots, \text{sat}(\nu^{l})\tr}\tr \in \RR^{n_{\phi}}
$$
The control input $u(k)$ and the input layers' vector $\nu_{\phi}$ can be expressed as a function of the state $x(k)$ and the last layers' output vector $\hat{\omega}_{\phi}(k)$

\begin{equation}
    \bmx{u(k)\\ \nu_{\phi}(k)} = \bmx{\begin{array}{c|cccc|c} 
        \0 & \0 & \dots & \0 & W^{l+1} & b^{l+1}\\
        \hline
        W^{1} & \0 & \dots & \0 & \0 & b^{1}\\
        \0 & W^{2} & \dots & \0 & \0 & b^{2}\\
        \vdots & \vdots & \ddots & \vdots & \vdots & \vdots\\
        \0 & \0 & \dots & W^{l} & \0 & b^{l}
    \end{array}}  \bmx{x(k) \\ \hat{\omega}_{\phi}(k) \\ 1}
\end{equation}

$$
    \bmx{u(k)\\ \nu_{\phi}(k)} = N  \bmx{x(k) \\ \hat{\omega}_{\phi}(k) \\ 1}\text{ and } N = \bmx{N_{ux} & N_{u \omega} & N_{ub} \\ N_{vx} & N_{v \omega} & N_{vb}}
$$

Then the equilibrium point $\left( x_{*}, u_{*}, \nu_{*}, \omega_{*} \right)$ is computed, 

\pagebreak
\printbibliography


\end{document}